\documentclass[12pt,upcase]{umlthesis}
\usepackage{lipsum}

% use fancyhdr, to enable page style stuff (below)
\usepackage{fancyhdr}
\setlength{\headheight}{15.2pt}
\renewcommand{\headrulewidth}{0pt}

\pagestyle{plain}

\begin{document}
\title{Two-Fluid Collisional Magnetohydrodynamic Modeling}
\author{Qusai Al Shidi}
\prevdegrees{B.S., University of Toledo (2008)}
\department{Department of Physics and Astronomy}
\degree{Doctor of Philosophy}
\degreemonth{May}
\degreeyear{2019}
\thesisdate{May 1, 2019}
\supervisor{Ofer Cohen}{Assistant Professor}
\maketitle

%%%%%%%%%%%%%%%%%%%%%%%%%%%%%%%%%%%%%%%%
\begin{abstract}
  This example document will utilize some of the features of \texttt{umlthesis.cls}.
\end{abstract}

%%%%%%%%%%%%%%%%%%%%%%%%%%%%%%%%%%%%%%%%
\begin{acknowledgments}
An acknowledgments page is optional.
\end{acknowledgments}

%%%%%%%%%%%%%%%%%%%%%%%%%%%%%%%%%%%%%%%%
\tableofcontents
\listoffigures
\listoftables

%%%%%%%%%%%%%%%%%%%%%%%%%%%%%%%%%%%%%%%%
%%%%%%%%%%%%%%%%%%%%%%%%%%%%%%%%%%%%%%%%
\chapter{Examples}
This chapter will show as much of the document class\footnote{This is a short footnote.} as possible. The footnotes\footnote{You'll find that this footnote is overly verbose and won't fit on a single line. If I have written the document class correctly, then the indentation will match the specification in the UML Thesis Guide.} are required to be indented in a particular way, for example.

\begin{quote}
  This is a paragraph quoted from somewhere else. It must be single-spaced and indented on both sides.
\end{quote}

In Table~\ref{tab:fruits} you can see what a table will look like. Refer to Figure~\ref{fig:square} to see a figure.

\begin{table}[h]\label{tab:fruits}
  \centering
  \caption[Comparison of fruits]{Only fruits of the same type may be compared safely.}
  \begin{tabular}{l|cc}
    & Apples & Oranges \\
    \hline
    Apples & yes & no \\
    Oranges & no & yes \\
  \end{tabular}
\end{table}

\lipsum[1]

\begin{figure}\label{fig:square}
  \centering
  \rule{2in}{2in}
  \caption{A black square.}
\end{figure}

\section{A Section}
This is what a main section heading looks like.

\lipsum[1]

\subsection{A Subsection}
Sub sections look like this.

\lipsum[1]

\subsubsection{A Sub-subsection (Don't go this deep!)}
Don't use sub-subsections.

\lipsum[1]

\section{Another Section}
This proves that the section numbering works.

%%%%%%%%%%%%%%%%%%%%%%%%%%%%%%%%%%%%%%%%
\chapter{Requirements}

\newcommand{\pkg}[1]{\textsf{#1}}

This document class requires \pkg{natbib},  \pkg{setspace}, and \pkg{tocloft}, which are probably already a part of your \LaTeX\ distribution.

%%%%%%%%%%%%%%%%%%%%%%%%%%%%%%%%%%%%%%%%
\nocite{*}
\bibliographystyle{plainnat}
\bibliography{example}

%%%%%%%%%%%%%%%%%%%%%%%%%%%%%%%%%%%%%%%%
\appendix
\chapter{Appendix Chapter}
\lipsum[2]

\end{document}
